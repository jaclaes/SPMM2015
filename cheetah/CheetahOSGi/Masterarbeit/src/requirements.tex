\chapter{Requirements}
\label{chapter:requirements}

This section analyzes the strengths and weaknesses of the mashup composer tools which could be
identified during the evaluation process and hence formulates the requirements for a software
product which eliminates the weaknesses and provides means for realizing the introduced scenario
(see Section \ref{sec:scenario}).

\begin{itemize}
  \item \textbf{Requirement 1: Simple User Interface}\newline The user interface that is provided by
  the evaluated tools can be mostly handled with the mouse and supports convenient drag and drop
  functionality. Hence, it does not need an extensive documentation and simple mashups can be designed
  very quickly by end-users.\newline Since user interfaces are one of the strengths of the mashup
  composers, they should be adopted and extended with the missing functionality, like grouping of
  blocks (cf. Requirement 5).
  \item \textbf{Requirement 2: Data Access and Processing}\newline As mashups and applications
  respectively most likely require data from local databases or from the web it is necessary to
  enable the access of a wide variety of data sources (from simple comma separated files
  over databases, over web services to live status information from an airplane).\newline In order
  to process the data which is received by the data access blocks it is necessary that each block can
  send data to and receive data from every other block. This enables the implementation of data
  conversion, aggregation and filtering components (cf. Requirement 3: Extensibility), which prepare
  the data for further receiving blocks and therewith reduce data incompatibilities between
  different building blocks.
  \item \textbf{Requirement 3: Extensibility}\newline The inspected mashup composers provide useful
  pre-built building blocks, but they are still not sufficient to cover every scenario, mashup or
  application respectively. Hence, the catalog of building blocks has to be extensible to provide
  displays for every kind of data and data access or processing functions for unfamiliar data
  formats.\newline Although creating a block requires programming skills, the process of
  implementing custom blocks should be as easy as possible and supported very well.
  \item \textbf{Requirement 4: Running multiple Instances of a Block}\newline This requirement
  is necessary to, for example, monitor multiple airplanes in parallel (see Section
  \ref{sec:scenario}), where the same building block is required multiple times. As IBM Mashup
  Center handles this problem quite well, this functionality should be adopted within the
  implementation of the solution.
  \item \textbf{Requirement 5: Grouping of Blocks}\newline This is one of the requirements which is
  not fulfilled to its full extend by any of the inspected tools, but constitutes a major requirement
  to fulfill the air ambulance scenario (e.g., when multiple flights have to be monitored at
  the same time).\newline Grouping of blocks should be enabled either in the form of multiple
  groups of components on a single user interface or in the form of multiple user interfaces, which are
  easily switchable.\newline This leads to a better solution for the air ambulance scenario.
  Having the grouping functionality, every block which displays information about the flight --
  like a map with the current position of the airplane or a table which provides information about
  crew and passengers -- can be arranged within a single group. In this way the information is
  encapsulated and the different flights can be distinguished more easily.
  \item \textbf{Requirement 6: Reusability of Groups}\newline The possibility of reusing groups is
  the next major requirement to realize the air ambulance scenario and to alleviate the process of
  developing more complex applications. Let us take the airplane monitor application which is
  responsible for displaying real-time information about multiple airplanes in parallel. The
  monitor of a single airplane consists of multiple different blocks and thus forms a group.
  Without the ability to reuse a group the blocks would have to be rearranged for each
  airplane.\newline Hence, applications which fulfill this requirement enable the saving of groups
  and thus also the arrangement of whole groups on the working area. As a consequence, groups can be
  handled like single building blocks and can be added to and removed from the mashup or application
  respectively by using simple drag and drop mechanisms.
  \item \textbf{Requirement 7: Hot Deployment and Life-cycle Management}\newline None of the
  evaluated tools support hot deployment and life-cycle management and hence are not applicable for the
  discussed scenario (see Section \ref{sec:scenario}).\newline The problem is, that once a mashup is
  designed, executed and used it cannot be adapted to user or role specific needs. Therefore, a
  framework is needed which supports the starting and stopping of components and groups at runtime
  as well as the installation of absolutely new components within the framework. The different
  states of a component, like installed, started and stopped must be handled by a life-cycle
  management system, which furthermore enables the hot deployment of components.
  \item \textbf{Requirement 8: Event Management}\newline As soon as a mashup becomes more complex
  and a considerable amount of data is involved, effective event handling needs to be in place.
  Imagine a data source block which has to send the whole data to each of the various interested
  data display blocks as soon as only a single data entry is updated.\newline A better solution for
  realizing this scenario would be to send an event as soon as the data is updated and if a display
  block is interested in the new data, it can poll the data from the data source block.
  Furthermore, the event management system can be extensively used to provide interested blocks
  with information about all kinds of events, like, for example, technical problems of an airplane.
  Therefore, it should be possible to declare blocks as event publishers as well as event handlers
  or listeners respectively.\newline As a consequence, each block has to provide human readable
  information about which events it can send to enable the receiver to react correctly.
  \item \textbf{Requirement 9: Logging}\newline As long as the programmers of software
  products cannot guarantee that their implementation is error-free, the integration of a logging mechanism is
  very helpful to get information about the errors that happened. This functionality should be
  integrated into every pre-built block as well as in the custom-built ones.\newline Furthermore, the
  logging mechanism can be used to conduct data and process mining methods.
\end{itemize}