\chapter{Summary}
\label{chapter:summary}

The evaluation of three mashup composers (see Chapter \ref{chapter:evaluation}), namely
``Microsoft Popfly'', ``Yahoo Pipes'' and ``IBM Mashup Center'' revealed, besides strengths like
the user interface, some major problems or shortcomings respectively, which are not solved or
implemented yet. Hence, requirements for a software product which integrates the identified
strengths and eliminates the weaknesses from the introduced evaluation criteria, are deduced (see
Chapter \ref{chapter:requirements}). Chapter \ref{chapter:osgi} introduces the Component Framework,
which is based on approved technologies and extends them with the required functionality to support
the identified requirements. Finally, Chapter \ref{chapter:applications} validates the introduced
framework by implementing a test application which can be seen as an interface to the Component
Framework and hence uses all its major functionalities.

\section{Conclusion}

To sum up, the Component Framework provides means to easily develop mashups and applications on the
basis of approved technologies like Java and OSGi. It eliminates most weaknesses of the evaluated
mashup composers by providing an extensive and further extensible grouping functionality of single
components, by making these groups reusable within other mashups, by integrating a life-cycle
management for single components as well as groups, by implementing an event management system
which enables a loosely coupled communication between components and by providing a logging
mechanism which provides a source for important feedback for the component developer and enables
data mining.

The implemented prototype application, which can be seen as an interface to the Component Framework,
finally provides methods to drag, drop and arrange components on a working area, to group components
within various forms of display, to connect components, in order to exchange, aggregate and process
data and to save and reuse the created projects.

Furthermore, a plug-in for Eclipse is provided, which alleviates the process of implementing custom
components and hence extend and enhance the framework or the prototype application respectively.

Hence, the developed solution provides better means than existing mashup composer tools to fulfill
the requirements of the introduced air ambulance scenario as well as of similar examples.

\section{Outlook}

As master theses are in most cases limited in time and resources that can be spent on it, the
Component Framework and the prototype application which were developed for this thesis indeed
fulfill the identified requirements, but still have great potentials to be further extended and
enhanced. A XML-based communication standard which is newly implemented or adapted from existing
service oriented technologies or the creation of a central service repository would be only two of
many interesting extensions.