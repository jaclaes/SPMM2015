\thispagestyle{empty}
\begin{center}
\huge \textbf{Abstract}
\end{center}
\vskip 2cm

\noindent Software products which are used within business are often utilized within different
business processes and by different users within different roles and hence should be flexible and
adaptable to different requirements. Therefore, this thesis evaluates so-called ``Mashup Composers''
which provide an interface to combine the nearly infinite number of data sources exhibited by the
world wide web. These tools introduce the necessary flexible service oriented approach and therewith
enable the combination and aggregation of multiple different services into a single application
without the need for any programming skills.

Yet, the evaluated software products are mainly applicable for monitoring purposes and lack some
essential functionalities, like hot deployment, grouping of blocks, event management and a logging
mechanism, to create powerful applications -- hence, the ``Component Framework'' was implemented.
Based on approved technologies like Java and OSGi this prototype framework provides an infrastructure
for developing, managing, displaying and connecting so-called components. To demonstrate the
feasibility of the framework a test application was implemented which provides an interface to this
framework and enables the composition of simple mashups and applications respectively by dragging and
dropping the various components onto a working area and connecting them.

\cleardoublepage